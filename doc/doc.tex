\documentclass{article}

\usepackage[utf8]{inputenc}
\usepackage[a4paper, total={6.5in, 10in}]{geometry}
\usepackage{amsmath}
\usepackage{xcolor}
\usepackage{hyperref}

\setlength{\parindent}{0pt}
\newcommand{\vxlisp}{\vspace*{12pt}}

\newcommand{\icd}[1]{\texttt{#1}}
\newcommand{\ccd}[1]{\colorbox{gray!15!white}{\texttt{#1}}}
\newcommand{\scd}[1]{
	\vspace*{5pt}

	\ccd{#1}
	\vspace*{5pt}
}
\newcommand{\nscd}[1]{\scd{\$ #1}}
\newcommand{\sscd}[1]{\scd{\$ sudo #1}}
\newcommand{\aptinstall}[1]{\sscd{apt install #1}}
\newcommand{\snapinstall}[1]{\sscd{snap install #1}}
\newcommand{\defos}{Ubuntu 20.10}
\newcommand{\psubinst}{possible installation using \icd{snap} on \defos}
\newcommand{\pubinst}{possible installation using \icd{apt} on \defos}
\newcommand{\pname}{brandy0}

\begin{document}

\author{Viktor Fukala}
\title{\pname: Fluid Flow Simulator}
\maketitle

\section{User Documentation}
\subsection{\pname}
\pname{} is a computer program that computes and displays the temporal evolution of the state of a fluid inside a two-dimentional container. To start the computation, the user must specifiy the parameters and boundary conditions the fluid should have and satisfy. The program also permits changing the parameters the computational model such as the sizes of the temporal and spacial steps during the simulation. When at least some states of the fuild are computed, the program allows the user to visualize them as a continuous video.

\subsection{Installation}

\pname{} can be distributed as a single executable file which dynamically links to libraries it depends on. The executable file itself can (in some cases) be directly downloaded from the project repository or it can be compiled from source. Either way, dependencies in the list that follows must be installed whenever \pname{} runs.

\subsubsection{Dependencies required at run time}\label{sec:runtimedeps}
For every dependency listed below, automatically include all its dependencies. (If using a package manager, they typically install automatically.)
\begin{itemize}
	\item gtkmm 3 (a C++ interface for GTK (GTK+) -- a GUI library)
		\begin{itemize}
			\item \pubinst{} (similarly for another linux package manager)
				\aptinstall{libgtkmm-3.0-1v5}
			\item Other versions, such as gtkmm 2 or gtkmm 4, are not compatible.
		\end{itemize}
	\item some ffmpeg libraries
		\begin{itemize}
			\item specifically \icd{libavcodec}, \icd{libavformat}, \icd{libavutil}, \icd{libswscale}
			\item \pubinst
				\sscd{add-apt-repository universe}
				\aptinstall{libavcodec58 libavformat58 libavutil56 libswscale5}
		\end{itemize}
\end{itemize}

\subsubsection{Additional dependencies required for compilation}\label{sec:compiledeps}
\begin{itemize}
	\item developer packages for the above run-time dependencies (unless the developer files were already included in the main package)
		\begin{itemize}
			\item \pubinst
				\vspace*{5pt}

				\ccd{\$ sudo apt install libgtkmm-3.0-dev libavcodec-dev libavformat-dev \textbackslash}\\
				\ccd{libavutil-dev libswscale-dev}
				\vspace*{5pt}
		\end{itemize}
	\item a C++17-compliant C++ compiler
		\begin{itemize}
			\item for example: possible installation of the GCC compiler using \icd{apt} on \defos
				\aptinstall{g++}
		\end{itemize}
	\item CMake version 3.16 or higher (a build automation tool)
		\begin{itemize}
			\item \psubinst{} (recommended)
				\snapinstall{cmake}
			\item \pubinst
				\aptinstall{cmake}
			\item Older versions of CMake may work fine too, but then it is necessary to edit the minimum required version in \icd{CMakeLists.txt}.
		\end{itemize}
	\item make (a build automation tool)
		\begin{itemize}
			\item \pubinst
				\aptinstall{make}
		\end{itemize}
	\item pkg-config (a build tool for fetching appropriate compiler flags)
		\begin{itemize}
			\item \pubinst
				\aptinstall{pkg-config}
		\end{itemize}
\end{itemize}

\subsubsection{Installation steps}
\underline{If you already have the executable file for \pname{}}:
\begin{enumerate}
	\item
		Install all dependencies mentioned in \ref{sec:runtimedeps}.
	\item
		Run the executable.
\end{enumerate}
\underline{If you need to compile the source}:
\begin{enumerate}
	\item
		Install all dependencies mentioned in \ref{sec:runtimedeps} and \ref{sec:compiledeps}.
	\item
		Clone the repository. In what follows, it is assumed that \icd{\$REPO} is the root directory of the repository.
	\item
		Decide whether you want a \emph{debug} or a \emph{release} build (or both, one executable for each). A release build should produce a more optimized executable, but it will probably make the compilation take longer and limit potential debugging options.
	\item
		Create a directory for the build. In what follows, \icd{\$REPO/build/release} will be the build directory.
		\nscd{mkdir -p \$REPO/build/release}
	\item
		Let CMake generate its build environment.
		\nscd{cd \$REPO/build/release}
		\nscd{cmake -DCMAKE\_BUILD\_TYPE=Release \$REPO/src} (replace \icd{Release} with \icd{Debug} if you want a debug build)
	\item
		Let CMake build the project.
		\nscd{cmake --build .} (run this still inside \icd{\$REPO/build/release})
	\item
		Run the executable. It has been generated at \icd{\$REPO/build/release/app/brandy0}.
\end{enumerate}

\subsection{General Principles}
\subsubsection{Physical Model}
The core object that the program attempts to approximate is the temporal evolution of the state of a fuild inside a container. In our context, the fluid is assumed to be incompressible and its state at any given moment is characterized by the velocity and the pressure of the fluid at every point in space. The computations of the program are based on a physical model which uniquely describes the temporal evolution of this state based on the following:
\begin{itemize}
	\item the shape of the enclosing container,
	\item the mechanical properties of the fluid,
	\item the boundary conditions at surface of the container, and
	\item the initial conditions (the initial state of the fluid).
\end{itemize}
With the exception of the initial conditions, the program allows the user to configure, to some extent, all of the above.

\subsubsection{Discretized Model}
In order for a computer to at least approximate the temporal evolution of the state of the fluid, the model must be discretized. The model must be discretized in both space and time. Therefore, the temporal evolution of the state is not represented as a map from a continuous interval but as a discrete sequence of finitely many frames, where each frame is an approximation of the state at one particular moment.
\vxlisp

Each frame then contains information about fluid pressure and velocity not at every point of the container but only at finitely many points of the container. In our case, these points form a uniform rectangular grid.

\subsubsection{Computational Realization}
The core of the program applies an iterative process which produces the next frame from the previous one. The specific formulas being applied are derived from differential equations governing the physical model.
\vxlisp

These frame-generating interations continue until the user actively stops the computation. To avoid running out of memory, all the frames are stored in a buffer with a set maximal number of frames. Whenever this limit is reached, every second frame is deleted from the buffer and every new frame is added to the buffer after twice as many iterations than how many were done for each frame before reaching this limit.

It must be noted that, not due any explicit error in either the computer implementation or the mathematical model but rather due to the very nature of the numerical model, the computation might collapse (\emph{diverge}). That means that especially for extereme values of some parameters, the program might produce degenerate results. However, when the results get degerate enough, the program automatically recognizes that and immeidately terminates further computations.
\vxlisp

Furthermore, even when the computations do not collapse, one should bear in mind that the results of the program are only an approximation of the real solution to the physical model, which in turn is only an approximation of the behavior of real-world fluids. Despite all this, the program helps visualize roughly what \emph{kind} of behavior the fluid is might exhibit under the given conditions.

\newcommand{\inull}{start interface}
\newcommand{\Inull}{Start Interface}
\newcommand{\ione}{configuration interface}
\newcommand{\Ione}{Configuration Interface}
\newcommand{\itwo}{simulation interface}
\newcommand{\Itwo}{Simulation Interface}
\newcommand{\iexv}{video export interface}
\newcommand{\Iexv}{Video Export Interface}

\subsection{Navigation}
The intended typical execution of the program can be logically divided into three phases.
\begin{enumerate}
	\item The \hyperref[sec:i0]{\inull}, which allows the user to initiate the configuration of a new simulation.
	\item The \hyperref[sec:i1]{\ione}, which allows the user to configure parameters of a simulation.
	\item The \hyperref[sec:i2]{\itwo}, which allows the user to play frames from the simulation that have already been computed, and which is being shown while the computation is running.
\end{enumerate}
The user may enter these interfaces in the above order, but they can also go back to the previous interface (unless they already are in the start interface)

\subsubsection{\Inull}\label{sec:i0}
The \inull{} consists of one main window with two buttons. The first button leaves the \inull{} and enters the \ione. The second button pops a second window with information about the program, but the main window remains open. Closing the popped up window with about information has no other effects. Closing the main window terminates the program.

% screenshot of i0 here (with the about box open)

\subsubsection{\Ione}\label{sec:i1}

The \ione{} consists of two windows by default. The main window allows the user to set various numerical parameters of the future simulation. (see [link here to description of parameters] for more detailed explanation of each parameter) Intially, the entry fields for simulation parameters contain some default values, which the user may subsequently change. In the main window, there are also buttons for going back to the \inull{} and for starting the simulation. It is only possible to start the simulation if no entry fields have been set to invalid values. (The presence of such an invalid value would be indicated by a red label next to the entry field.)
\vxlisp

The second window in the \ione{} allows the user to configure the physical shape of the container. That includes the physical dimensions of the enclosing rectangle (its width and height) and the shape, position and size of the solid obstacles inside this rectangle. The user can select the shape of the obstacle they want to add and then specify the position and size of the obstacle by clicking on the grid. For more information about specific shapes, see [link here to description of shapes]. In the top right part of the window, the user can undo the addition of shapes, redo the addition they have already undone, or clear remove all shapes (without the option to then undo this removal).
\vxlisp

Closing the main window terminates the application. Closing the shape configuration window enables a button in the main window that reopens it.

\subsubsection{\Itwo}\label{sec:i2}

The \itwo{} consists of one main window. This main window contains a drawing area which at every moment, displays one of the computed frames of the simulation. There is a control panel below the drawing area. Parts of the control panel include a button to terminate the simulation and return to \ione{}, a section  ...
% need to divide this into more sections (section per Gtk::Frame in control panel)
% perhaps use itemize
% need to link to part of a screenshot to make this at least a bit comprehensible

\section{Technical Documentation}

\end{document}
